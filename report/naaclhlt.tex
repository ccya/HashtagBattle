%
% ML final project writeup
%

\documentclass[11pt,letterpaper]{article}
\usepackage{naaclhlt2010}
\usepackage{listings}
\usepackage{varioref}
\usepackage{times}
\usepackage{latexsym}
\usepackage{color}
\setlength\titlebox{6.5cm}    % Expanding the titlebox

\title{Hashtag Battle Project}

\author{Shuya Chu\\
  {\tt csy.ellie@gmail.com}
}

\date{\today}

\begin{document}
\maketitle


\section{Design}




\section{Main Working}
\subsection{Feature Generation}

\subsubsection{Data Pre-processing}
\paragraph{Generating Location Feature}

\section{Results}
\label{sec:results}

\begin{center}
   \begin{table}[h]
    \begin{tabular}{ | l | l  | p{2cm} |}
    \hline
     & SVM(RBF) & SVM(linear) \\ \hline
    correct prediction& 106 & 106  \\ \hline
    test sample size& 150 &150  \\ \hline
    accuracy & 70.67\% &70.67\%\\ \hline
    \end{tabular}
    \caption{Comparison of SVM with different kernels}
    \end{table}
\end{center}

\begin{center}
   \begin{table}[h]
    \begin{tabular}{ | l | l  | p{2cm} |}
    \hline
     & SVM(RBF) & SVM(linear) \\ \hline
    correct prediction& 106 & 106  \\ \hline
    test sample size& 150 &150  \\ \hline
    accuracy & 70.67\% &70.67\%\\ \hline
    \end{tabular}
    \caption{Comparison of SVM after PCA}
    \end{table}
\end{center}

It's a little surprising that the prediction remains the same after a reduction of 25\% of original feature space. When one-third feature is left for learning, the number of correct prediction for linear kernel decreases. We think that the sparsity of feature space may cause such result. 

To see whether SVM outperforms other learning algorithms, we test data on Naive Bayes, logistic regress with L1 penalty ,stochastic gradient descent methods . Among all,  naive Bayes method has relative poor performance. What's unexpected is that stochastic gradient algorithm's performance is slight better than SVM in this case. In the following table, we show prediction accuracy with different classification algorithms under full feature space against reduced feature space. By looking at Logistic Regression and Naive Bayes, feature reduction indeed improve performance and filtering out noise in data. 
\begin{center}
   \begin{table}[h]
    \begin{tabular}{ | l | l  | p{2cm} |}
    \hline
     & Logistic Reg & Naive Bayes \\ \hline
    correct prediction& 105 & 99  \\ \hline
    test sample size& 150 &150  \\ \hline
    accuracy & 70\% &66\%\\ \hline
    \end{tabular}
    \caption{ Logistic Reg and Naive Bayes with full feature space}
    \end{table}
\end{center}

\begin{center}
   \begin{table}[h]
    \begin{tabular}{ | l | l  | p{2cm} |}
    \hline
     & Logistic Reg & Naive Bayes \\ \hline
    correct prediction& 106 & 102  \\ \hline
    test sample size& 150 &150  \\ \hline
    accuracy & 70\% &68\%\\ \hline
    \end{tabular}
    \caption{Logistic Reg and Naive Bayes with PCA}
    \end{table}
\end{center}

\begin{center}
   \begin{table}[t]
    \begin{tabular}{ | l | l  | p{2cm} |}
    \hline
     & SGD L1& SGD L2 \\ \hline
    correct prediction& 99 & 107  \\ \hline
    test sample size& 150 &150  \\ \hline
    accuracy & 66\% &71.3\%\\ \hline
    \end{tabular}
    \caption{SGD with L1 and L2 regularization in full space}
    \end{table}
\end{center}



\begin{center}
   \begin{table}
    \begin{tabular}{ | l | l  | p{2cm} |}
    \hline
     & SGD L1& SGD L2 \\ \hline
    correct prediction& 104 & 106  \\ \hline
    test sample size& 150 &150  \\ \hline
    accuracy & 66\% &71.3\%\\ \hline
    \end{tabular}
    \caption{SGD with L1 and L2 regularization in reduced space}
    \end{table}
\end{center}

\vspace{-2em}
\section{Comparison with proposal}
Compared with what we have mentioned in the proposal, we are able to achieve most of them, except the one that turn model into an application. Now, lets check whether items in Milestones of proposal is completed in this report. 
\vspace{-1em}
\begin{description}
  \item[Must achieve] \hfill \\
  For "feature construction", "Feature Selection","Apply ML Algorithm", we do complete in section ~\ref{sec:methods}
  \vspace{-1em}
  \item[Expected to achieve] \hfill \\
  For accuracy analysis, we do complete it in section ~\ref{sec:results}
  \vspace{-1em}
  \item[Third] \hfill \\
  We didn't turn it into application successfully, but we instead we come up with an idea of using SVR to give more precise score to reflect one's preference. 
\end{description}



\vspace{-2em}

\begin{thebibliography}{}
\bibitem{MB} M Hu and B Liu. 2014. Mining opinion features in customer reviews. \emph{In Proceedings of the National Conference on Artificial \ldots}  
\bibitem{GRC} G A Miller, R Beckwith, and C Fellbaum. 1990. Introduction to word net: An on-line lexical database. \emph{International journal \ldots} (1990)
\bibitem{FP} F. Pedregosa, G. Varoquaux, A. Gramfort, V. Michel, B. Thirion, O. Grisel, M. Blondel, P. Prettenhofer, R. Weiss, V. Dubourg, J. Vanderplas, A. Pas- sos, D. Cournapeau, M. Brucher, M. Perrot, and E. Duchesnay. 2011. Scikit-learn: Machine Learning in Python. \emph{Journal of Machine Learning Research} 12 (2011), 2825?2830.
\bibitem{BVJ} Bryan Hood, Victor Hwang and Jennifer King. 2013. Inferring Future Business Attention.


\end{thebibliography}

\section*{Source code}
\lstset{language=Python,basicstyle=\footnotesize}
\begin{lstlisting}
from sklearn import svm
from sklearn.naive_bayes import GaussianNB
from sklearn.linear_model import SGDClassifier
from sklearn.linear_model import LogisticRegression
f = open('../result/features_train')
g = open('../result/features_dev')
lines = f.readlines()
dev_lines = g.readlines()
Y = []
X = []
X_dev = []
Y_dev = []
for each in lines[1:]:
	wl = each.split('|')
	instance = []
	for i in xrange(len(wl)):
		if i == 0:
			label = wl[i][1]
			Y.append(int(label))
		elif i >= 1 and i<=50:
			instance.append(int(wl[i].rstrip('\n')))
	#print wl[1]
	#Y.append(wl[0][1])
	#instance.append(wl[1])
	X.append(instance)
clf = svm.SVC()
rbf_svm = svm.SVC(kernel='linear')
clf.fit(X,Y)
rbf_svm.fit(X,Y)
gnb = GaussianNB()
sgd = SGDClassifier(loss="hinge", penalty="l2")
gnb.fit(X,Y)
sgd.fit(X,Y)
clf_l1_LR = LogisticRegression(penalty='l1', tol=0.01)
clf_l1_LR.fit(X, Y)


for each in dev_lines[1:]:
	wl = each.split('|')
	instance = []
	for i in xrange(len(wl)):
		if i == 0:
			label = wl[i][1]
			Y_dev.append(int(label))
		elif i >= 1 and i<=50:
			instance.append(int(wl[i].rstrip('\n')))
	X_dev.append(instance)

result = clf.predict(X_dev)
result_rbf = rbf_svm.predict(X_dev)
result_nb = gnb.predict(X_dev)
result_lg = clf_l1_LR.predict(X_dev)
result_sgd = sgd.predict(X_dev)
right = 0
right = 0
for i in xrange(len(result)):
	if result[i] == Y_dev[i]:
		right += 1 
print right
print len(result)

right = 0
for i in xrange(len(result_rbf)):
	if result_rbf[i] == Y_dev[i]:
		right += 1 
print right
print len(result_rbf)

right = 0
for i in xrange(len(result_nb)):
	# print str(Y_dev[i]) + str(result[i])
	if result_nb[i] == Y_dev[i]:
		right += 1 
print right
print len(result_nb)


right = 0
for i in xrange(len(result_lg)):
	# print str(Y_dev[i]) + str(result[i])
	if result_lg[i] == Y_dev[i]:
		right += 1 
print right
print len(result_lg)

right = 0
for i in xrange(len(result_sgd)):
	# print str(Y_dev[i]) + str(result[i])
	if result_sgd[i] == Y_dev[i]:
		right += 1 
print right
print len(result_sgd)


\end{lstlisting}


\clearpage
\lstset{language=Python,basicstyle=\footnotesize}
\begin{lstlisting}
import operator
# f = open('../result/sorted_keywords')
g = open('../result/keywordpoint')
points = g.readlines()
# put the points for keywords in dictionary
point_dic = {}
weight_dic = {}
for each in points:
	pairs = each.rstrip('\n').split('|')
	keyword  = pairs[0]
	weight =  pairs[1] + pairs[2]
	point_dic.setdefault(keyword, (pairs[1],pairs[2]))
	weight_dic.setdefault(keyword, weight)
sorted_weight_dic = sorted(weight_dic.iteritems(), key=operator.itemgetter(1),reverse = True)
# print sorted_weight_dic[0]
# set the subset size
size = 200
# get the subset keywords and corresponding point
result = {}
for i in xrange(size):
	word = sorted_weight_dic[i][0]
	if word in point_dic:
		point = point_dic[word]
		# print point
		result.setdefault(word,point)
	else:
		result.setdefault(word,(0,0))
k = open('../result/subset','w+')
for each in result:
	value = each + ":" + str(result[each])
	k.write(value + '\n')
\end{lstlisting}

\clearpage
\lstset{language=Python,basicstyle=\footnotesize}
\begin{lstlisting}
import re
import json
f = open('../result/keywordpoint_train')
g = open('../result/subset')
# h = open('../result/sampled_bid')
h = open('../result/not_sampled_bid')
m = open('../result/loc_feature')
# k = open('../result/features_train','w+')
k = open('../result/features_dev','w+')

#Get labels
bids = h.readlines()
bidlist = []
labels = {}
for each in bids:
	bid = each.split(':')[0]
	label = each.split(':')[1].rstrip('\n')
	labels[bid] = label
	bidlist.append(bid)

# Get loc features
loc_fea = {}
locinfo = m.readlines()
for each in locinfo:
	bid = each.split(' ')[0]
	fear = each.split(' ')[1].rstrip('\n')
	loc_fea[bid] = fear


# Get dict, key is busid, value is keyword,count pairs. i.e. a dict of dict
dic = {}
put = False
lines = f.readlines()
for line in lines:
	if re.match('business_id',line):
		bid = line.split(" ")[1].rstrip('\n')
		if bid in bidlist:
		# if bid not in bidlist:
			dic[bid] = {}
			put = True
		else:
			put = False
	else:
		keyword = line.split('|')[0]
		p = line.split('|')[1]
		n = line.split('|')[2].rstrip('\n')
		if put:
			if keyword in dic[bid]:
				dic[bid][keyword] = (int(dic[bid][keyword][0])+int(p),int(dic[bid][keyword][1])+int(n))
			else:
				dic[bid][keyword] = (p,n)

#Start to create feature

k.write('label loc ')
for i in xrange(200):
	k.write('K' + str(i) +'P K'+str(i) + 'N ')
kl = []
keywords = g.readlines()
for each in keywords:
	kl.append(each.rstrip('\n').split(":")[0])

for each_bid in dic:
	k.write('\n' +labels[each_bid])
	# Y.append(labels[each_bid])
	k.write('|' + loc_fea[each_bid])
	instance = [loc_fea[each_bid]]
	for each_kw in kl:
		found = False
		for key in dic[each_bid]:
			if key in each_kw:
				value = '|' +str(dic[each_bid][key][0]) + '|' +str(dic[each_bid][key][1])
				k.write(value)
\end{lstlisting}

\clearpage
\lstset{language=Python,basicstyle=\footnotesize}
\begin{lstlisting}
import re
import sys


################################################################################
#------------------Find keyword for all review of ONE usr-----------------------
f = open('../result/chunck')
g = open('../result/keywords','w+')
k = open('../result/adj_keyword','w+')
keywords_adj = {}
lines = f.readlines()
ladj = []

for i in xrange(len(lines)):
	if re.match("business_id", lines[i]):
		k.write(lines[i])
	if re.match("  \(NP",lines[i]):
		keyword = re.sub(r'NP|/NNP|/NNS|/NN|[()]|', '',lines[i]).rstrip('\n')[3:]
		if len(keyword)!=0:
			# g.write(keyword+'\n')
			k.write(keyword + ":")
			j = i+1
			find = True
			ladj = []
			while(find):
				if len(lines[j])<4:
					if j == len(lines) - 1:
						find = False
					else:
						j = j+1
				else:
					if re.search(r'\/JJ',lines[j][-4:]) is not None:
						adj = re.sub(r'/JJ|/JJS|/JJR','',lines[j]).rstrip('\n')[2:]
						ladj.append(adj)
						if keyword in keywords_adj:
							keywords_adj[keyword].append(adj)
						else:
							keywords_adj.setdefault(keyword,[adj])
						j = j+1
					elif re.match("  \(NP",lines[j]):
						find = False
						k.write(",".join(ladj)+ " \n")
					elif j == len(lines)-1:
						find = False
						k.write(",".join(ladj)+ " \n")
					elif re.match('business_id',lines[j]):
						find = False
						k.write(",".join(ladj)+ " \n")
					else:
						j = j+1
for each in keywords_adj:
	g.write(each + ':' + (',').join(keywords_adj[each])+ '\n')
\end{lstlisting}



\clearpage
\lstset{language=Python,basicstyle=\footnotesize}
\begin{lstlisting}

import json
import math
import collections
f = open('../result/bus_loc')
g = open('../data/yelp_academic_dataset_business.json')
k = open('../result/loc_feature','w+')
loc = {}
alldist = []
lines = f.readlines()
latitude = 0
longtitude = 0
for each in lines:
	longtitude += float(each.split(":")[1].split(" ")[1])
	latitude += float(each.split(":")[2].split(" ")[1])
center_lat = latitude/774
center_long = longtitude/774

max = 0.0
preferd = []
for each in lines:
	lon = float(each.split(":")[1].split(" ")[1])
	lat = float(each.split(":")[2].split(" ")[1])
	dist = math.sqrt((lat - center_lat)**2 + (lon - center_long)**2)
	preferd.append(dist)
sorted_preferd = sorted(preferd)

length = len(sorted_preferd)
medium_1 = 0.0
if not length % 2:
	medium_1 = (sorted_preferd[length/2]+sorted_preferd[length/2-1])/2.0
else:
	medium_1 = sorted_preferd[length/2]
print medium_1


for line in g:
	data = json.loads(line)
	lat = data["latitude"]
	lon = data["longitude"]
	dist = math.sqrt((lat - center_lat)**2 + (lon - center_long)**2)
	alldist.append(dist)
	loc[data["business_id"]] = dist

sorted_dist = sorted(alldist)

# length = len(sorted_dist)
# medium = 0.0
# if not length % 2:
# 	medium = (sorted_dist[length/2]+sorted_dist[length/2-1])/2.0
# else:
# 	medium = sorted_dist[length/2]
# print medium

fea_loc = {}
for each in loc:
	if loc[each]<medium_1:
		fea_loc[each] = 0
	else:
		fea_loc[each] = 1
result = collections.OrderedDict(sorted(fea_loc.items()))
for each in result:
	k.write(each+ " " + str(fea_loc[each])+"\n")
\end{lstlisting}
\end{document}


